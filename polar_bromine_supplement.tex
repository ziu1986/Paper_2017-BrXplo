\documentclass[manuscript]{copernicus}
\graphicspath{{pictures/}}          % graphics
\begin{document}
\clearpage
\setcounter{page}{1}

\section*{Supplement}
\subsection*{S.1 Heterogeneous reactions involving bromine used in the model}
\begin{reaction}
  \chem{HOBr}+\chem{HBr}\rightarrow\chem{Br_2}+\chem{H_2O}
\end{reaction}
\begin{reaction}
  \chem{BrNO_3}+\chem{H_2O}\rightarrow\chem{HOBr}+\chem{HNO_3}
\end{reaction}
\begin{reaction}
  \chem{ClNO_3}+\chem{HBr}\rightarrow\chem{BrCl}+\chem{HNO_3}
\end{reaction}
\begin{reaction}
  \chem{BrNO_3}+\chem{HCl}\rightarrow\chem{BrCl}+\chem{HNO_3}
\end{reaction}
\begin{reaction}
  \chem{HOCl}+\chem{HBr}\rightarrow\chem{BrCl}+\chem{H_2O}
\end{reaction}
\begin{reaction}
  \chem{HOBr}+\chem{HCl}\rightarrow\chem{BrCl}+\chem{H_2O}
\end{reaction}
%
\subsection*{S.2 Nassi-Shneiderman diagram}
\begin{center}
\includegraphics[height=0.90\textheight]{pictures/sup_airsnow_emission}
\end{center}
%
\subsection*{S.3 Excerpt from \texttt{onemis} namelist}
\begin{verbatim}
&CTRL
! ###########################################################################
! ### GAS
! ###########################################################################
<list of gas emissions>
EMIS_TYPE(9) = 'AirSnow'
! ###########################################################################
! ### AEROSOL
! ###########################################################################
<list of aerosol emissions>
/
&CPL_IMPORT
<list of import fields>
! ### AirSnow ###############################################################
! Dry deposition fluxes
imp_ddepflux_HOBR  = 'ddep_gp', 'ddepflux_HOBr'
imp_ddepflux_BrNO3 = 'ddep_gp', 'ddepflux_BrNO3'                          
imp_ddepflux_HBr   = 'ddep_gp', 'ddepflux_HBr'
imp_ddepflux_O3    = 'ddep_gp', 'ddepflux_O3'
! Multi-year sea ice fraction
imp_sic_multi_year = 'import_grid', 'airsnow_mysic'
/
&CPL 
L_LG       = F  ! EMISSIONS FOR LAGRANGIAN TRACERS
<list of flux to tracer conversion>
! ...Br2 and BrCl from airsnow
F2T(13) = 'snow_air_flux_br2',  'Br2:M=2,SC=1.0',  'Br2:M=2,SC=1.0',
F2T(14) = 'snow_air_flux_brcl', 'BrCl:M=2,SC=1.0', 'BrCl:M=2,SC=1.0',
/
&CTRL_AirSnow
! Default values according to Toyota et al. 2011 parametrization
r_crit_temp      = -15 ! Critical temperature [deg celsius]
r_sun_theta_crit = 85  ! Critical sun zenith angle [deg]
! Efficiency of bromine release due to ozone deposit ('dark','sunlit','land')
r_trigger_1      = 0.001_dp, 0.075_dp, 0.0_dp
/
\end{verbatim}
%
\subsection*{S.4 \chem{BrO} total vertical column density zonal mean}
\subsubsection*{S.4.1 GOME}
\begin{center}
\includegraphics[width=0.9\textwidth, clip, trim={2cm 0.75cm 2cm 1.75cm}]{pictures/GOME_BrO_tot_zonal_mean}
\subsubsection*{S.4.2 EMAC (BrXplo\_mysic)}
\includegraphics[width=0.9\textwidth, clip, trim={2cm 0.75cm 2cm 1.75cm}]{pictures/EMAC_BrO_tot_zonal_mean}
\end{center}
%
\subsection*{S.5 Sensitivity study with increased surface resistance of \chem{O_3}}
\subsubsection*{S.5.1 Northern hemisphere}
\begin{center}
\includegraphics[width=1\textwidth, angle=90]{pictures/rs_surface_ozone}
\end{center}
%
\subsubsection*{S.5.2 Southern hemisphere}
\begin{center}
\includegraphics[width=1\textwidth, angle=90]{pictures/rs_surface_ozone_sh}
\end{center}
%
\subsection*{S.6 Temporal correlation of modeled surface \chem{O_3} with observation}
\subsubsection*{S.6.1 EMAC (BrXplo\_ref)}
\begin{center}
\includegraphics[height=0.45\textheight]{pictures/surface_ozone_corr_ref}
\end{center}
%
\subsubsection*{S.6.2 EMAC (BrXplo\_mysic)}
\begin{center}
\includegraphics[height=0.45\textheight]{pictures/surface_ozone_corr_mysic}
\end{center}
%
\subsubsection*{S.6.3 EMAC with increased surface resistance}
\begin{center}
\includegraphics[height=0.45\textheight]{pictures/surface_ozone_corr_mysic_rsnow}
\end{center}


\end{document}
